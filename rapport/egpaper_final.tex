\documentclass[10pt,twocolumn,letterpaper]{article}

\usepackage{cvpr}
\usepackage{times}
\usepackage{epsfig}
\usepackage{graphicx}
\usepackage{amsmath}
\usepackage{amssymb}\usepackage[utf8x]{inputenc}
\usepackage[T1]{fontenc}
\usepackage[francais]{babel}

% Include other packages here, before hyperref.

% If you comment hyperref and then uncomment it, you should delete
% egpaper.aux before re-running latex.  (Or just hit 'q' on the first latex
% run, let it finish, and you should be clear).
\usepackage[breaklinks=true,bookmarks=false]{hyperref}

\cvprfinalcopy % *** Uncomment this line for the final submission

\def\cvprPaperID{****} % *** Enter the CVPR Paper ID here
\def\httilde{\mbox{\tt\raisebox{-.5ex}{\symbol{126}}}}

% Pages are numbered in submission mode, and unnumbered in camera-ready
%\ifcvprfinal\pagestyle{empty}\fi
\setcounter{page}{4321}
\begin{document}

%%%%%%%%% TITLE
\title{Rapport de Deep Learning}

\author{Antoine Aubay\\
	CentraleSupélec\\
	% For a paper whose authors are all at the same institution,
	% omit the following lines up until the closing ``}''.
	% Additional authors and addresses can be added with ``\and'',
	% just like the second author.
	% To save space, use either the email address or home page, not both
	\and
	Anais Kayaturan\\
	CentraleSupélec\\
}

\maketitle
%\thispagestyle{empty}

%%%%%%%%% ABSTRACT
\begin{abstract}
   Bla bla.
\end{abstract}

%%%%%%%%% BODY TEXT
\section{Etat de l'art}

Nous présentons dans un premier temps un état de l'art détaillant les enjeux associés à l'analyse des MEG et les différentes techniques mises en place jusque là.

%-------------------------------------------------------------------------
\subsection{Principes de la MEG}

Les techniques d'électroencéphalographie (EEG) et de magnétoencéphalographie (MEG) sont des méthodes complémentaires de mesure de l'activité neurologique du cerveau. Lorsque le cerveau effectue une action particulière, cela déclenche l'activation d'ensembles de neurones caractéristiques de cette action qui alors émettent des potentiels d'action de manière synchrone. Mesurer ces potentiels d'action et réussir à les caractériser permet donc de comprendre ce que fait le cerveau.

L'EEG mesure les potentiels d'action émis par les neurones tandis que la MEG mesure les champs magnétiques résultant. Les deux techniques ont de bonnes résolutions temporelles (de l'ordre de la milliseconde), mais une mauvaise résolution spatiale (de l'ordre du centimètre). L'idéal est de combiner les deux techniques afin d'obtenir les résultats les plus précis possibles.

Au terme d'une séances de mesure avec une EEG ou une MEG, on obtient une série temporelle pour chaque capteur.

\subsection{Deep Learning appliqué aux EEG et aux MEG}

Si l'intérêt pour le Deep Learning est croissant dans le domaine de l'imagerie médicale, notamment sur les IRM, la recherche reste assez peu aboutie sur les EEG et inexistante sur les MEG. Les recherches se concentrent sur les EEG car c'est une méthode qui existe dans de nombreux laboratoires de par son faible coût, contrairement aux MEG qui constituent un investissement important.

Les problématiques liées aux EEG sont connues et plusieurs techniques ont été employées pour tenter de pallier à ces problèmes. Les EEG et les MEG mesurent des données provenant de la même source (l'activité du cerveau) et de nature proche (activité électrique vs magnétique). Le postulat est que les problématiques connues liées aux EEG se retrouveront sur les MEG et justifient donc que nous nous intéressions dans un premier temps aux EEG.

\subsubsection{Problématiques liées aux EEG}
Voici les différentes problématiques connues liées aux EEG :

\begin{itemize}
	\item Faible ratio signal/bruit : Le bruit est très fortement présent sur les EEG. Il peut provenir de diverses sources : battement cardiaque, clignement des yeux, activité électrique aux alentours (passage d'un métro par exemple), etc...
	\item Peu de données disponibles : Les EEG et les MEG sont des données très coûteuses à produire (plusieurs centaines d'euros pour une séance de travail avec un EEG pour un patient) et les laboratoires les partagent peu. Il faut donc utiliser des techniques d'augmentation de données pour parvenir à des résultats convenables.
	\item Differences de signal inter-patients car l'activité cérébrale change d'un individu à l'autre
	\item Différences de signal intra-patients : Déformations du signal dans le temps dues aux conditions d'acquisition du signal (déplacement des électrodes sur le cuir chevelu au cours de l'expérience)
\end{itemize}
Pour ces raisons, l'application du deep learning sur les EEG donne des résultats de l'ordre de 50 pourcents pour le moment.

\subsection{Le jeu de données}
Le jeu de données que nous avons choisi est

{\bf There will be no extra page charges for
  CVPR 2017.}


\begin{quote}
\begin{center}
    An analysis of the frobnicatable foo filter.
\end{center}

   In this paper we present a performance analysis of our
   previous paper [1], and show it to be inferior to all
   previously known methods.  Why the previous paper was
   accepted without this analysis is beyond me.

   [1] Removed for blind review
\end{quote}


\subsection{Miscellaneous}

\noindent
Compare the following:\\
\begin{tabular}{ll}
 \verb'$conf_a$' &  $conf_a$ \\
 \verb'$\mathit{conf}_a$' & $\mathit{conf}_a$
\end{tabular}\\


\begin{figure*}
\begin{center}
\fbox{\rule{0pt}{2in} \rule{.9\linewidth}{0pt}}
\end{center}
   \caption{Example of a short caption, which should be centered.}
\label{fig:short}
\end{figure*}

%------------------------------------------------------------------------
\section{Formatting your paper}

All text must be in a two-column format. The total allowable width of the
text area is $6\frac78$ inches (17.5 cm) wide by $8\frac78$ inches (22.54
cm) high. Columns are to be $3\frac14$ inches (8.25 cm) wide, with a
$\frac{5}{16}$ inch (0.8 cm) space between them. The main title (on the
first page) should begin 1.0 inch (2.54 cm) from the top edge of the
page. The second and following pages should begin 1.0 inch (2.54 cm) from
the top edge. On all pages, the bottom margin should be 1-1/8 inches (2.86
cm) from the bottom edge of the page for $8.5 \times 11$-inch paper; for A4
paper, approximately 1-5/8 inches (4.13 cm) from the bottom edge of the
page.



Figure and table captions should be 9-point Roman type as in
Figures~\ref{fig:onecol} and~\ref{fig:short}.  Short captions should be centred.

\noindent Callouts should be 9-point Helvetica, non-boldface type.
Initially capitalize only the first word of section titles and first-,
second-, and third-order headings.


{\small
\bibliographystyle{unsrt}
\bibliography{egbib}
}

\end{document}
